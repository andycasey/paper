\documentclass[modern, letterpaper]{aastex631}

\usepackage{microtype}
\usepackage{amsmath}

\newcommand{\package}[1]{\textsf{#1}}
\newcommand{\project}[1]{\textsl{#1}}
\newcommand{\acronym}[1]{{\small{#1}}}
\newcommand{\Gaia}{\project{Gaia}}
\newcommand{\dd}{\ensuremath{\,\mathrm{d}}}

\shorttitle{Unresolved spectroscopic binaries with \Gaia\ I}
\shortauthors{Foreman-Mackey et al.}

\begin{document}

\title{An open source scientific article}

\title{Unresolved spectroscopic binaries with \Gaia\ I: \\
	Detection \& orbital characterization using EDR3 radial velocity errors}

\correspondingauthor{Daniel~Foreman-Mackey}
\email{dforeman-mackey@flatironinstitute.org}

\author[0000-0002-9328-5652]{Daniel Foreman-Mackey}
\affiliation{Center for Computational Astrophysics, Flatiron Institute, 162 5th Ave, New York, NY 10010, USA}

\author{Quadry~Chance}
\affiliation{Center for Computational Astrophysics, Flatiron Institute, 162 5th Ave, New York, NY 10010, USA}
\affiliation{University of Florida, 211 Bryant Space Science Center, Gainesville, FL 32611, USA}

\author{Andrew~R.~Casey}
\affiliation{School of Physics \& Astronomy, Monash University, Wellington Road, Clayton 3800, Victoria, Australia}

\author{Others TBD}
\noaffiliation{}

\begin{abstract}
	The Early Data Release 3 (EDR3) from the \Gaia\ Mission includes measurements of the radial velocities (RVs) of more than 7 million targets with the goal of providing a detailed view of the phase space of the Milky Way.
	Many of these stars are, in fact, members of unresolved multiple star systems and the effects of these orbits are generally seen as a source of contamination in the RV catalog.
	In this paper, we demonstrate that the published RV error estimates can provide evidence for the existence of an unresolved stellar companion and, under the assumption of a gravitationally-bound binary system, place constraints its orbital parameters.
	This paper makes three key contributions:
	\emph{(1)} We build a flexible data-driven model to estimate \Gaia's intrinsic per-transit RV uncertainty as a function of color and apparent magnitude.
	\emph{(2)} Using these estimates, we develop a procedure for detecting likely unresolved multiple star systems, and apply this procedure to all targets observed at least 3 times in EDR3.
	\emph{(3)} Finally, we derive a physical model for constraining the orbital parameters of a binary system that would be consistent with the observed RV error and number of RV measurements.
	These parameters are generally only weakly constrained, except for the RV semi-amplitude, which can typically be estimated with a precision of $\sim PPP\%$.
	To demonstrate the reliability of this method and quantify its completeness, we apply our method to simulated data and catalogs of previously known spectroscopic binaries.
	We expect that this value-added catalog will be useful for population studies of binary star systems, target selection for precise RV surveys, vetting exoplanet discoveries, and other applications.
\end{abstract}

\keywords{Astrostatistics (1882) --- Binary stars (154) --- Radial velocity (1332)}

\section{Introduction} \label{sec:intro}

By the end of the \Gaia\ Mission, it will discover SOME LARGE NUMBER of exoplanet and multiple star systems based on time resolved astrometry and radial velocities of many targets.
In the meantime, we only have static measurements of the radial velocity, but it turns out that there is still information in the existing public facing catalog to place constraints on the orbital parameters of multiple systems using the \Gaia\ data.
It has been previously demonstrated that the published radial velocity and astrometric ``errors'' can be used a proxy for multiplicity or data quality.
In this paper, however, we demonstrate that it is possible to use the reported radial velocity errors and a probabilistic model to place constraints on the radial velocity amplitude and, in some cases, some other properties of the orbit.
These measurements are useful for many applications, including \emph{(a)} discovering black holes, \emph{(b)} vetting transiting exoplanet discoveries, \emph{(c)} measuring the masses of a large sample of eclipsing binaries, \emph{(d)} quantify the binary fraction across the H--R diagram, and \emph{(e)} informing constraints on exoplanet formation and evolution theory, to name a few.

The fundamental idea of this paper is that even though the \Gaia\ catalog only includes a single RV measurement and its error, we can still make calibrated probabilistic inferences about the RV time series.
In particular, we know the number of RV observations that were used to measure the RV and its uncertainty.
Similarly, we can estimate the expected per-transit RV uncertainty based on the distribution of targets of similar color and magnitude.
Using these auxillary values, we can compute the expected distribution of RV error, and compare this to the measured value.
It has been previously identified that the RV error measurements scale with RV semi-amplitude (IS THERE A CITATION; add a figure), but below we take this one step further and derive an interpretable, physical, probabilistic model for this effect that allows us to directly infer the semi-amplitude, and weak constraints on the other orbital parameters.

In the following sections, we derive and discuss the components of this model, validate and characterize the performance using simulated data, and compare to archival binary surveys.
To start, we build a hierarchical probabilistic model to estimate the per-transit RV measurement uncertainty of the \Gaia\ survey as a function of apparent magnitude and observed color.

\section{The basics}
\label{sect:basics}

For each RV target, the \Gaia\ EDR3 catalog reports the median RV, an estimate of the error on this median, and the number of individual RV measurements (called ``transits'') that were used in this estimate.
In our analysis, we ignore the median RV and instead only consider the RV error and number of transits.
The RV error $\epsilon_n$ for a target $n$ is computed as \citep{Katz19}
\begin{eqnarray}
	{\epsilon_n}^2 = \left(\sqrt{\frac{\pi}{2\,T_n}}\,s_n\right)^2 + 0.11^2
\end{eqnarray}
where
\begin{eqnarray}
	\label{eq:sample-var}
	{s_n}^2 = \frac{1}{T_n-1}\sum_{i=1}^{T_n} \left(v_{n,i} - \bar{v}_n\right)^2
\end{eqnarray}
is the sample variance of the individual RV measurements $v_{n,t}$.
In \autoref{eq:sample-var}, $t=1,\cdots,T_n$ indexes the $T_n$ transits for target $n$ and
\begin{eqnarray}
	\bar{v}_n = \frac{1}{T_n}\sum_{i=1}^{T_n} v_{n,i}
\end{eqnarray}
is the mean RV.
In all that follows, it is much simpler to reason about the sample variance ${s_n}^2$ instead of the error on the median ${\epsilon}_n$, but that isn't a problem because we can compute
\begin{eqnarray}
	{s_n}^2 = \frac{2\,T_n}{\pi}\left({\epsilon_n}^2 - 0.11^2\right)
\end{eqnarray}
for any target, using only values available in the catalog: $\epsilon_n$ and $T_n$.

Now, for a given target $n$, in the absence of any excess radial velocity signal, and if we know it's per-transit radial velocity measurement uncertainty $\sigma_n$, we would expect the quantity
\begin{eqnarray}
	{\xi_n}^2 &=& \frac{(T_n - 1)\,s_n^2}{{\sigma_n}^2}
	\label{eq:chi-sq-samp}
\end{eqnarray}
to be chi-squared distributed with $T_n - 1$ degrees of freedom.
Based on this argument, we can identify targets that are outliers with respect to this expectation.
\autoref{fig:infer_sigma:data} illustrates this idea using a simulated catalog of \Gaia\ RV targets.
Here, we generate a sample of \Gaia\ RV measurements for binary star systems with orbital parameters drawn from a simple population model, and compare the distribution of measured ${s_n}^2$ values to those that we would measure in the absence of binary companions.
In \autoref{fig:infer_sigma:data}, both distributions peak near the (known) measurement uncertainty $\sigma^2$---a fact that we will use in the following section to \emph{infer} $\sigma^2$---but the distribution of binary systems shows a tail to larger values of $s^2$.

We can even go one step further, and derive the expected sampling distribution for ${s_n}^2$ under some model for the excess RV signal.
For example, in our case, we want to infer the orbital properties of any binary systems based on the observed ${s_n}^2$.
In this case, our model for the expected RV time series is (CITE L\&F)
\begin{eqnarray}
	\mu_n(t;\,\theta_n) &=& v_{0,n} + K_n\,\left[\cos(f_n(t) + \omega_n) + e_n\,\cos \omega_n\right] \quad,
	\label{eq:rv-model}
\end{eqnarray}
but the following derivation can be used more generally.
We can evaluate \autoref{eq:rv-model}, at observation times $\{t_{n,i}\}_{i=1}^{T_n}$ and orbital parameters $\theta_n=\{K_n,\,e_n,\,\omega_n,\,\ldots\}$.
Of course, in our case, we don't actually \emph{know} the observation times, so we will need to marginalize those out below, but for now, let's proceed as if we do know them.
For a given set of parameters, it can be shown (see \autoref{app:ncx2}) that the sampling distribution for our scalar
\begin{eqnarray}
	{\xi_n}^2 &=& \frac{(T_n - 1)\,s_n^2}{{\sigma_n}^2}
	\label{eq:chi-sq-samp2}
\end{eqnarray}
is a non-central $\chi^2$ distribution with $T_n$ degrees of freedom, and non-centrality parameter
\begin{eqnarray}
	\lambda &=& \sum_{i=1}^{T_n} \left[\frac{\mu_n(t_i;\,\theta_n) - \bar{\mu}_n}{\sigma_n}\right]^2
	\label{eq:ncx2-nc-param}
\end{eqnarray}
where
\begin{eqnarray}
	\bar{\mu}_n &=& \frac{1}{T_n}\sum_{i=1}^{T_n} \mu_n(t_i;\,\theta_n)
\end{eqnarray}
is the sample mean of the model evaluated at the observation times.
This sampling distribution defines a likelihood function for the parameters $\theta_n$, that we will form the foundation for the probabilistic models built in the following sections.

In all the above, we assumed that the per-observation RV measurement uncertainty $\sigma_n$ was known.
This is not, however, a number that we have access to.
Instead, in the next section we derive a hierarchical model for determining a calibrated data-driven estimate of $\sigma_n$, as a function of color and apparent magnitude, based on the published EDR3 catalog.

\begin{figure}
	\begin{centering}
		\includegraphics[width=0.8\textwidth]{figures/infer_sigma_data.pdf}
		\caption{The distribution of RV sample variance ${s}^2$ for a simulated population of binary stars observed by \Gaia\ (unfilled histogram), compared to the distribution that would have been observed in the absence of companions (shaded histogram).
			The peak of each distribution is determined by the per-measurement RV uncertainty $\sigma^2$, but the catalog of binaries exhibits a heavy tail to large values of $s^2$, which we will use to detect these systems in the real \Gaia\ EDR3 catalog.}
		\label{fig:infer_sigma:data}
	\end{centering}
\end{figure}

\section{Estimating Gaia's per-transit RV measurement uncertainty}\label{sec:noise}

A key element of our analysis is that we must be able to make, for each target $n$ in our sample, a reasonably accurate estimate of the per-transit RV measurement uncertainty, called $\sigma_n$ in the previous section.
The \Gaia\ catalog does not include this value, but as we will demonstrate, it is possible to make an estimate of $\sigma_n$ for each target $n$, using the data available in the catalog.
In practice, we will build a data-driven model to infer $\sigma$ in small bins in observed $G_\mathrm{BP} - G_\mathrm{RP}$ color and apparent $G$-band magnitude $m_\mathrm{G}$, making the simplifying assumption that $\sigma$ is approximately constant at a given color and magnitude.
There are other factors, such as distance and stellar evolutionary state, that will affect the measurement uncertainty for a given target, but we find that including these axes has a negligible effect on the inferred $\sigma$, and the added complexity is not required for the level of precision required here.

To set up this model, we create a fine grid in color and magnitude, and then for each cell of this grid, we infer $\sigma_n$ independently of the other bins.
In each bin, we assume that all the targets are unresolved binaries with unknown RV semi-amplitude, and we fit for the orbital parameters of each target and the shared RV measurement uncertainty, simultaneously.
In principle, we could use the likelihood function defined in the previous section---and our assumption that the value of $\sigma_n$ is constant within a small enough color--magnitude bin---to simultaneously infer $\sigma_n$, $\theta_n$ for a sample of \Gaia\ targets.
This proposal is made somewhat more complicated by the fact that we also don't know the observation times $t_{n,i}$ and need to marginalize over them.
We discuss this in more detail in SOMESECTION, but here we will make the further simplifying assumptions to define a tractable effective model.
We restrict ourselves (for now) to circular orbits and assume that the orbit of each binary is well-sampled.
This means that we can approximate the non-centrality parameter as
\begin{eqnarray}
	\lambda_{\mathrm{eff},n} &=& \frac{T_n\,{K_n}^2}{2\,\pi\,{\sigma_n}^2}\int_{0}^{2\,\pi} \cos^2\phi \dd\phi = \frac{T_n\,{K_n}^2}{2\,{\sigma_n}^2} \quad,
	\label{eq:lambda-eff}
\end{eqnarray}
where $K_n$ is the unknown RV semi-amplitude of the orbit of target $n$.\footnote{Some intution for the factor of $T_n$ in \autoref{eq:lambda-eff} can be developed by recognizing that $\lambda$ is defined in \ref{eq:ncx2-nc-param} as $T_n$ times the sample variance of the model, evaluated at the observed times.}

Given this definition for $\lambda_{n,\mathrm{eff}}$, we can define the probabilistic model
\begin{eqnarray}
	p(\sigma_n,\,\{K_n\}\,|\,\{T_n,\,{s_n}^2\}) \propto
	p(\sigma_n)\,\prod_{n=1}^{N} p(K_n)\,p(T_n,\,{s_n}^2\,|\,\sigma_n,\,K_n)
\end{eqnarray}
where the sampling distribution $p(T_n,\,{s_n}^2\,|\,\sigma_n,\,K_n)$ is a non-central $\chi^2$ distribution with $T_n - 1$ degrees of freedom and non-centrality parameter $\lambda_{\mathrm{eff},n}$ defined by \autoref{eq:lambda-eff}.\footnote{We note that since the sampling distribution is defined for $\xi_n^2$, which is a function of $\sigma_n$, we must (and do!) include the Jacobian in our definition of $p(T_n,\,{s_n}^2\,|\,\sigma_n,\,K_n)$, with respect to the usual definition of the non-central $\chi^2$ probability density.}
For $p(\sigma_n)$ and $p(K_n)$, we assume the following broad, weakly-informative priors
\begin{eqnarray}
	\ln \sigma_n &\sim& \mathcal{N}(0, 10) \nonumber\\
	\ln K_n &\sim& \mathcal{N}(0, 10) \quad.
\end{eqnarray}
Given these definitions, we implement this model using the \package{numpyro} probabilistic programming language and estimate the joint posterior using variational inference.
SOME MORE WORDS ABOUT VI.

We validate this procedure by generating a simulated \Gaia\ catalog for a sample of unresolved binaries (a large fraction of which have undetectable amplitude) with known orbital parameters and RV measurement uncertainty.
This catalog is shown in \autoref{fig:infer_sigma:data}, and \autoref{fig:infer_sigma:results} demonstrates that this procedure provides an estimate of the true RV uncertainty and, to some extent, the injected RV semi-amplitude for each target.
This is not a trivial experiment, because the procedure used to simulate the catalog was not consistent with the simplifying modeling assumptions.
In particular, the simulated values for $\sigma_n$ were generated from a log-normal distribution $\ln\sigma_n \sim \mathcal{N}(\ln\sigma_0,\,0.1)$.
Furthermore, the simulated RV error values ${s_n}^2$ were generated by procedurally sampling observation times and orbital phases, injecting noise, and computing the sample variance.
In most cases, the orbits are far from well-sampled.
This helps to demonstrate that the accuracy of our inference procedure is not significantly affected by our simplifying assumptions.

\begin{figure}
	\begin{centering}
		\includegraphics[width=0.8\textwidth]{figures/infer_sigma.pdf}
		\caption{For the simulated \Gaia\ catalog from \autoref{fig:infer_sigma:data}, this figure shows the recovered RV semi-amplitude for each simulated target, as a function of its true semi-amplitude.
			The dashed horizontal line shows the mean RV measurement uncertainty used to simulate the data, compared to the inferred value shown in blue.}
		\label{fig:infer_sigma:results}
	\end{centering}
\end{figure}

In practice, the values for $\sigma_n$ inferred using the above procedure are somewhat noisy.
To regularize our estimates in each bin with more than 1000 targets, we repeat the inference for 5 different, randomly-selected, 1000-target samples, and average the resulting estimates of $\sigma_n$.
We also perform some light smoothing in color and magnitude to remove some remaining sample noise in bins with few targets.
Our final estimate of the per-transit RV uncertainty $\sigma_n$ as a function of color and magnitude is shown in the color map in \autoref{fig:noise_model}.
The shaded sections of that figure indicate regions where there were fewer than 50 targets per bin.
For targets outside the well sampled regime, our estimate of $\sigma_n$ is extrapolated, and should be considered suspect.
\autoref{fig:noise_model} shows that the \Gaia\ RV spectrograph can reach a precision of better than $300\,\mathrm{m\,s^{-1}}$ for the brightest targets, with more typical values being of order $1\,\mathrm{km\,s^{-1}}$.

\begin{figure}
	\begin{centering}
		\includegraphics[width=0.8\textwidth]{figures/noise_model.pdf}
		\caption{The estimated per-transit \Gaia\ RV measurement uncertainty as a function of color and magnitude.
			The shaded region indicates areas where there were fewer than 50 targets per bin, so this estimate is extrapolated to these regions.}
		\label{fig:noise_model}
	\end{centering}
\end{figure}

\autoref{fig:sigma_cmd} shows the mean RV measurement uncertainty across the color--magnitude diagram, where we have estimated the distance of each source using the inverse parallax.
In this figure, there is some shot noise visible around the edges where the target density drops, but there is not otherwise a huge amount of structure.
DO WE WANT TO SAY ANYTHING ELSE?

\begin{figure}
	\begin{centering}
		\includegraphics[width=0.8\textwidth]{figures/sigma_cmd.pdf}
		\caption{The average per-transit RV uncertainty across the color--magnitude diagram.
			The contours indicate the density of targets in this same space to guide the eye.
			Most of the structure in this diagram can be seen near the edges where the number of samples is small.
			WHAT ABOUT THOSE BLUE BOIS?}
		\label{fig:sigma_cmd}
	\end{centering}
\end{figure}


\section{Binary probabilities}

One side effect of inferring the per-transit radial velocity uncertainty for all sources in \Gaia\ is that we also get (for free!) an estimate of the probability that the measured RV error is consistent with what we would expect if the individual RV measurements were Gaussian distributed with this uncertainty.
This is some sort of ``outlier'' probability, but we can use it as a proxy for binarity in the absence of systematic, or other un-accounted for effects.
We explore the limitations of such an assumption in the following sections, but it is useful for now to discuss the math for the idealized case here.

As discussed in \autoref{sect:basics}, if the RV measurements for a given \Gaia\ target $n$ includes only Gaussian noise with variance ${\sigma_n}^2$, the sampling distribution for the measured, normalized RV variance ${\xi_n}^2 = (T_n - 1)\,{s_n}^2 / {\sigma_n}^2$ is $\chi^2$ with $T_n - 1$ degrees of freedom.
Therefore, we can compute the $p$-value for this null hypothesis as
\begin{eqnarray}
	\mathrm{Pr}(\xi^2 > {\xi_n}^2\,|\,\mathrm{null}) &=& \int_{\xi_n^2}^\infty \chi^2 (\xi^2;\,T_n-1) \dd \xi^2 \quad.
\end{eqnarray}
A detailed discussion of the interpretation of $p$-values is beyond the scope of this paper, but it's useful to recognize that this value quantifies the frequency with which we would expect to mislabel a single star as a binary, given all the assumptions of our model.
% This is a useful number since, it can be used to construct a catalog with a specific expected false positive rate.

We construct a fiducial binary catalog by selecting all targets with a $p$-value below some threshold $P_0$.
If our Gaussian model is correct and our estimates of the per-transit measurement uncertainty is accurate, then we would expect targets without companions to pass threshold with probability $P_0$.
In other words, setting $P_0 = 0.001$, we would expect 1 in 1,000 single targets to pass our cut and be incorrectly labelled as a binary.
This number does not directly map to the false positive rate, since that will also depend on the true binary fraction, however, it gives us a tuning parameter that can trade off precision and recall in our selection.

\autoref{fig:p_value_dist} shows the observed distribution of $p$-values across the full catalog.
Our null hypothesis here is that every target is a single star with only Gaussian noise with known variance as estimated in \autoref{sec:noise} and, if this hypothesis were correct, we would expect the $p$-values to be uniformly distributed between zero and one (as indicated by the orange dashed histogram in \autoref{fig:p_value_dist}).
The fact that the observed distribution is significantly different from this expectation shows that there are a significant number of outliers.

After computing the $p$-value for every target in our catalog, we can use these values to identify targets that are likely multi-star systems or other kinds of outliers.
\autoref{fig:binary_fraction_cmd} shows the fraction of targets with $p < 0.001$ as a function of their location in the color--magnitude diagram.
In the absence of other sources of noise\footnote{We discuss these other noise sources in SOME SECTION.}, this can be interpreted to first order as the binary fraction, but it is important to remember that this is conditioned on the sensitivity of our detection procedure, an effect that we discuss below.
With this interpretation, this figure shows some expected structures such as an equal-mass main sequence and an increased binary fraction at higher mass.

\begin{figure}
	\begin{centering}
		\includegraphics[width=0.8\textwidth]{figures/p_value_dist.pdf}
		\caption{The distribution of $p$-values across the full sample. In the absence of outliers, we would expect to find the distribution shown as a dashed orange line.}
		\label{fig:p_value_dist}
	\end{centering}
\end{figure}

\begin{figure}
	\begin{centering}
		\includegraphics[width=0.8\textwidth]{figures/binary_fraction_cmd.pdf}
		\caption{Like \autoref{fig:sigma_cmd}, but this time the color bar shows the fraction of targets with a $p$-value less than 0.001. This is an estimate of the fraction of \Gaia\ targets that are detectable multiple star systems as a function of color and magnitude. In other words, this can (with the appropriate caveats) be interpreted as the binary fraction across the color--magnitude diagram. Under this interpretation, the equal-mass binary main sequence is clearly visible, and there is an increased binary fraction DESCRIBE THIS.}
		\label{fig:binary_fraction_cmd}
	\end{centering}
\end{figure}



\section{Completeness \& reliability}

We would like to characterize the efficiency with which our method detects astrophysical signals from companions.
In other words we want to estimate the \emph{completeness} of our survey.
To do this, we use simulated injection and recovery tests where we generate simulated data with a known signal and attempt to recover it with our pipeline.
In particular, we will focus on characterizing the survey completeness as a function of the radial velocity measurement uncertainty and semi-amplitude, integrated over all of the other dimensions.
There are other parameters that affect the detection thresholds including, most importantly, the number of RV transits, but these are second order effects and we will marginalize over these effects using realistic distributions.

We use an approximate broken power-law model for the number of radial velocity transits:
\begin{eqnarray}
	\label{eq:nb-transits}
	p(T_n) &\propto& \left\{\begin{array}{ll}
		1                                  & T_n < T_\mathrm{break} \\
		(T_n / T_\mathrm{break})^{-\alpha} & \mathrm{otherwise}
	\end{array} \right.
\end{eqnarray}
where the parameters are set to approximately match the observed distribution.
These parameters are set as follows: the breakpoint is $T_n = 12$ and the power-law index is $\alpha = 6$.
As demonstrated by SOMEFIGURE, this approximate distribution is a reasonably good fit.

Otherwise we sample the other parameters period, RV semi-amplitude, phase, eccentricity, argument of periasteron, and per-transit measurement uncertainty from the simple distributions listed in SOMETABLE.

Then we simulate an RV time series with $T_n$ observations sampled from \autoref{eq:nb-transits} and compute the sample variance.
Then for consistency, we compute the floored and scaled ``error'' $\epsilon$ used to construct the \Gaia\ catalog (SOMEEQUATION) and, like the real catalog, mark any with $\epsilon > 20\,\mathrm{km/s}$ as non-detections.
Using this catalog, we evaluate the $p$-value for each simulated target, and then estimate its RV semi-amplitude.
Figure SOMEFIGURE shows the recovery rate as a function of RV semi-amplitude and per-transit measurement uncertainty.
At small semi-amplitude, the injected signals are not recovered since the variance is too small to distinguish from the measurement noise.
At large semi-amplitude, the signals are not recovered because the catalog is censored at large RV ``error'' ($\epsilon > 20\,\mathrm{km/s}$).
Figure SOMEOTHERFIGURE shows the inferred semi-amplitude for these simulations as a function of the true value.

These measurements don't fully capture all the limitations of this inference, but we will try to address some of those below by comparing to other binary surveys.

% \begin{figure}
% 	\plotone{figures/completeness.pdf}
% 	\caption{The completeness of our binary detection pipeline calculated using a simulated catalog of binary star systems.
% 		The color bar indicates the fraction of simulated binaries that are recovered---where recovery is defined as the system having a $p$-value of less than 0.001.
% 		\emph{(left)} The recovery rate of binary targets as a function of the RV measurement uncertainty and the true RV semi-amplitude of the simulated orbit.
% 		At fixed measurement uncertainty, the recovery rate increases with semi-amplitude because the signal-to-noise of the error measurement increases.
% 		The recovery rate also decreases for large semi-amplitudes because the \Gaia\ processing pipeline discards any measurements where the final RV error is larger than 20~km/s, an effect that is included in our simulated catalog.
% 		\emph{(right)} The same color bar as the right-hand panel, but this time plotted as a function of the number of RV transits.
% 		Perhaps unsurprisingly, the recovery rate is significantly higher for targets with larger numbers of observations.
% 		% It's worth noting that since this figure is integrated over a simulated distribution of transit numbers, the detailed shape depends on the actual distribution.
% 		\label{fig:completeness}}
% \end{figure}

\begin{figure}
	\begin{centering}
		\includegraphics[width=\textwidth]{figures/completeness.pdf}
		\caption{A figure.}
		\label{fig:completeness}
	\end{centering}
\end{figure}



\section{A model for the radial velocity error}

Apparently \Gaia\ reports RV error $\epsilon$ defined as:
\begin{eqnarray}
	\epsilon^2 = \left(\sqrt{\frac{\pi}{2\,N}}\,s\right)^2 + 0.11^2
\end{eqnarray}
where
\begin{eqnarray}
	s^2 = \frac{1}{N-1}\sum_{n=1}^N \left(v_n - \bar{v}\right)^2
\end{eqnarray}
is what we actually want.

We need to compute a probability density for the $s^2$ parameter above conditioned on a set of $N$ noisy observations of an RV orbit (with some given parameters).
I'm going to model the noise as Gaussian with a known, constant variance $\sigma^2$, but it's \emph{possible} (although probably tricky) that we could relax some of that.

Next, the key realization is that if we have a set of $N$ Gaussian random variables $X_n \sim \mathcal{N}\left(\mu_n,\,\sigma^2\right)$ (with known, but different, means $\mu_n$, in our case given by the RV orbit evaluated at $t_n$), then the random variable $Y = \sum X_n^2$ will be distributed as a noncentral chi-squared.
To use this to our advantage, we first argue that the argument in the variance sum $v_n - \bar{v}$ will be a Gaussian random variable
\begin{eqnarray}
	v_n - \bar{v} \sim \mathcal{N}\left(\mu_n - \bar{\mu},\,\sigma^2\right)
\end{eqnarray}
where $\mu_n$ is the RV orbit \emph{model} at time $t_n$.
This isn't totally obvious since the $v_n$ and $\bar{v}$ distributions aren't necessarily independent, but I think it's ok because [insert relevant math here] (this derivation will look similar to the sampling distirbution for the sample variance calculations so I should double check there, but this definitely works in practice).

Then we're where we need to be: $(N - 1)\,s^2$, our quantity of interest, is now the sum of squares of $N$ Gaussian random variables.
Following the Wikipedia page for the noncentral chi-squared distribution\footnote{\url{https://en.wikipedia.org/wiki/Noncentral_chi-squared_distribution}}, we need to rescale our parameters to have unit variance:
\begin{eqnarray}
	\frac{v_n - \bar{v}}{\sigma} \sim \mathcal{N}\left(\frac{v_n - \bar{\mu}}{\sigma},\,1\right)
\end{eqnarray}
where we have used the assumption that $\sigma$ is known and constant.
I think that we could allow non-constant $\sigma$ but that would result in a "generalized chi-squared" distribution which has no closed form\footnote{\url{https://en.wikipedia.org/wiki/Generalized_chi-squared_distribution}}.

Then the parameters of our noncentral chi-squared will be

\begin{eqnarray}
	k = N
\end{eqnarray}

and

\begin{eqnarray}
	\lambda = \sum \frac{N\,(\mu_n - \bar{\mu})^2}{(1 + N)\,\sigma^2}
\end{eqnarray}

And the parameter that is thus distributed is:

\begin{eqnarray}
	\xi^2 = \frac{(N - 1)\,s^2}{\sigma^2}
\end{eqnarray}

That was a bit ugly.
Let's see if this works with a simulation.
We'll simulate a single simple RV curve and then compute the distribution of $s^2$ over many different realizations of the noise.

\section{Bulk radial velocity inference}

Finally we can combine the measurement uncertainty estimates from SOME SECTION with the

\section{Artisanal radial velocities}

\section{Validation}

\section{Discussion}

This is a paper.
See \autoref{fig:noise_model}.


% \begin{figure}
% 	\plotone{figures/recovered.pdf}
% 	\caption{The semi-amplitude estimated from only the RV error and number of transits for a suite of simulated binary orbits, as a function of the true simulated semi-amplitude.
% 		The points are colored by the eccentricity of the orbit showing a weak trend. MORE HERE?
% 		\label{fig:recovered}}
% \end{figure}

% \begin{figure}
% 	\plotone{figures/sb9.pdf}
% 	\caption{This is a cool figure. \label{fig:sb9}}
% \end{figure}

% \begin{figure}
% 	\plotone{figures/apogee-gold.pdf}
% 	\caption{This is a cool figure. \label{fig:apogee-gold}}
% \end{figure}

\begin{acknowledgments}
	The authors would like to thank the Astronomical Data Group at Flatiron for listening to every iteration of this project and for providing great feedback every step of the way.
\end{acknowledgments}

\vspace{5mm}
\facilities{
	\project{APOGEE},
	\project{Gaia}
}
\software{
	\package{AstroPy} \citep{Astropy2013, Astropy2018},
	\package{JAX} \citep{Bradbury2018},
	\package{NumPy} \citep{Harris2020},
	\package{Matplotlib} \citep{Hunter2007},
	\package{SciPy} \citep{Virtanen2020}
}

\appendix

\section{The non-central $\chi^2$ distribution}
\label{app:ncx2}

\bibliography{bib}{}
\bibliographystyle{aasjournal}

\end{document}
